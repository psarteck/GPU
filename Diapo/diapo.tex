\documentclass[10pt]{beamer}

% \usetheme{Montpellier}

\usepackage[T1]{fontenc}
\usepackage[utf8]{inputenc}  
\usefonttheme{professionalfonts}

\setbeamersize{text margin left=30pt, text margin right=30pt}

\setbeamercolor{section in head/foot}{fg=blue}

\title{Présentation projet final CPU/ GPU}
\author{Matthieu PETIT \\ Djemsay MORVAN \\ Ulysse CAROMEL}
\setbeamertemplate{section in toc}[sections numbered]
\setbeamertemplate{navigation symbols}{}
\setbeamertemplate{itemize items}[square]

\begin{document}
\begin{frame}
    \includegraphics[width=0.2\textwidth]{../Images/univ-rennes1.png}
    \rule{\linewidth}{0.3mm} \\[0.4cm]
    \titlepage
    \rule{\linewidth}{0.3mm} \\[0.4cm]
\end{frame}

\begin{frame}
    \frametitle[Table des matières]{Table des matières}
    \tableofcontents
\end{frame}

\section{Les différentes méthodes d'intégration}

\begin{frame}
    \frametitle{Méthodes d'intégration}  
    \tableofcontents[currentsection]  
\end{frame}

\subsection{Méthode de Simpson}


\begin{frame}
    {\fontsize{4}{5}\selectfont Texte en police très petite}

    \frametitle{Méthode de Simpson}

    \begin{itemize}
        \item Division de l'intervalle $[a, b]$ en sous-intervalles de largeur égale $h = \frac{b - a}{n}$, $n$ étant pair.
        
        \item Approximation sur chaque sous-intervalle $[x_i, x_{i+2}]$ :
            \[
            \int_{x_i}^{x_{i+2}} f(x) \,dx \approx \frac{h}{3} \left[ f(x_i) + 4f(x_{i+1}) + f(x_{i+2}) \right]
            \]
        
        \item Estimation finale de l'intégrale :
            \begin{align*}
                \int_{a}^{b} f(x) \,dx \approx \frac{h}{3} \Big [ f(a) + 4f(x_1) + 2f(x_2) &+ \ldots + 2f(x_{n-2})  \\
                &+ 4f(x_{n-1}) + f(b) \Big ]
            \end{align*}
    \end{itemize}
    
\end{frame}


\subsection{Méthode de Gauss 2D}
\begin{frame}
    \frametitle{Quadrature Gaussienne 2D}
    \footnotesize
    \begin{itemize}
        \item Estimation numérique de l'intégrale d'une fonction $f(x, y)$ sur une région bidimensionnelle définie par $a \leq x \leq b$ et $c \leq y \leq d$.
        
        \item Utilise un ensemble de points et de poids associés pour approximer l'intégrale.
        
        \item Formule générale :
            \begin{align*}
            \iint_{R} f(x, y) \,dx\,dy \approx \sum_{i=1}^{n} \sum_{j=1}^{m} w_{ij} \cdot f(x_i, y_j)
            \end{align*}
        
        \item Formule spécifique pour un quadrilatère :
            \begin{align*}
            \iint_{R} f(x, y) \,dx\,dy \approx & \sum_{i=1}^{n} \sum_{j=1}^{m} w_{ij} \cdot f\Bigg(\frac{1}{2}(1 + \xi_i)x\\
             &+ \frac{1}{2}(1 - \xi_i)y, \frac{1}{2}(1 + \eta_j)x + \frac{1}{2}(1 - \eta_j)y\Bigg)
            \end{align*}
        
        \item Offre une précision supérieure à la quadrature de Gauss unidimensionnelle.
    \end{itemize}
\end{frame}

\subsection{Méthode de Runge-Kutta}
\begin{frame}
    \frametitle{Méthode de Runge-Kutta (RK4) pour les EDO}
    \small
    \begin{itemize}
        \item Technique numérique pour résoudre des équations différentielles ordinaires (EDO).
        
        \item Méthode de Runge-Kutta d'ordre 4 (RK4) : équilibre entre précision et complexité.
        
        \item Forme générale d'une EDO du premier ordre :
            \[
            \frac{dy}{dt} = f(t, y)
            \]
        
        \item Étapes de la méthode RK4 :
            \begin{align*}
            k_1 &= h \cdot f(t_n, y_n) \\
            k_2 &= h \cdot f(t_n + \frac{h}{2}, y_n + \frac{k_1}{2}) \\
            k_3 &= h \cdot f(t_n + \frac{h}{2}, y_n + \frac{k_2}{2}) \\
            k_4 &= h \cdot f(t_n + h, y_n + k_3)
            \end{align*}
        
        \item Mise à jour de la solution à chaque pas :
            \begin{align*}
             y_{n+1} = y_n + \frac{1}{6}(k_1 + 2k_2 + 2k_3 + k_4)
            \end{align*}
        
        \item Offre une meilleure précision que des méthodes de pas fixe plus simples, largement utilisée pour sa robustesse et polyvalence.
    \end{itemize}
\end{frame}

\subsection{Méthode de Monte-Carlo}
\begin{frame}
    \frametitle{Méthode de Monte Carlo pour l'intégration}
    \small
    \begin{itemize}
        \item Utilisée pour estimer numériquement des intégrales complexes, notamment dans des espaces multidimensionnels.
        
        \item Fondée sur des principes probabilistes, associant le hasard à des calculs numériques.
        
        \item Estimation d'une valeur en utilisant des échantillons aléatoires dans un domaine donné.
        
        \item Génération de points aléatoires dans le domaine $D$.
        
        \item Évaluation de la fonction $f(x, y)$ pour chaque point généré.
        
        \item Calcul de la moyenne pondérée des valeurs de $f(x, y)$ pour les points générés, multipliée par la mesure de $D$.
    \end{itemize}

    Formule d'estimation de Monte Carlo pour l'intégrale :
    $$I \approx A \cdot \frac{1}{N} \sum_{i=1}^{N} f(x_i, y_i)$$

    Avantages : Flexibilité, capacité à traiter des problèmes complexes en dimensions élevées. Précision dépend du nombre de points aléatoires générés.

\end{frame}

\subsection{Comparaison des méthodes}
% \begin{frame}
%     \small
%     \frametitle{Comparaison des Méthodes Numériques}
%     \begin{itemize}
%         \item \textbf{Quadrature Gaussienne 2D (Gauss 2D):}
%             \begin{itemize}
%                 \item \textbf{Avantages :} Précision élevée, adaptée aux intégrales sur des domaines bidimensionnels complexes.
%                 \item \textbf{Inconvénients :} Nécessite la connaissance de la fonction à intégrer, plus de calculs.
%             \end{itemize}

%         \item \textbf{Méthode de Runge-Kutta (RK4) pour les EDO :}
%             \begin{itemize}
%                 \item \textbf{Avantages :} Équilibre entre précision et complexité, adaptée aux équations différentielles ordinaires (EDO).
%                 \item \textbf{Inconvénients :} Nécessite plus de calculs que des méthodes de pas fixe plus simples.
%             \end{itemize}

%         \item \textbf{Utilisation de la Police \ttfamily :}
%             \begin{itemize}
%                 \item \textbf{Avantages :} Contrôle direct de la police de machine à écrire dans tout le document.
%                 \item \textbf{Inconvénients :} Dépend de la disponibilité de la police choisie sur le système.
%             \end{itemize}

%         \item \textbf{Méthode de Monte Carlo pour l'Intégration :}
%             \begin{itemize}
%                 \item \textbf{Avantages :} Flexibilité, traitement de problèmes complexes en dimensions élevées, approche probabiliste.
%                 \item \textbf{Inconvénients :} Précision dépend du nombre de points aléatoires générés, nécessite plus de ressources.
%             \end{itemize}
%     \end{itemize}
% \end{frame}



\begin{frame}
    \frametitle{Comparaison des méthodes numériques}

    \begin{columns}[T] % Alignement des colonnes en haut
        \begin{column}{.45\textwidth}
            \underline{\textbf{Runge-Kutta (RK4)}}
            \begin{itemize}
                \item[+] Précision élevée.
                \item[+] Adaptée aux EDO.
                \item[-] Plus de calculs.
            \end{itemize}
            \vspace{10pt}
            \underline{\textbf{Gaussienne 2D}}
            \begin{itemize}
                \item[+] Haute précision.
                \item[+] Adaptée aux intégrales bidimensionnelles.
                \item[-] Plus complexe.
            \end{itemize}
        \end{column}
        
        \begin{column}{.45\textwidth}
            \underline{\textbf{Simpson}}
            \begin{itemize}
                \item[+] Simple à mettre en œuvre.
                \item[+] Bonne précision.
                \item[-] Limité aux formes simples.
            \end{itemize}
            \vspace{10pt}
            \underline{\textbf{Monte Carlo}}
            \begin{itemize}
                \item[+] Grande flexibilité.
                \item[+] Adaptée aux problèmes complexes.
                \item[-] Précision dépend du nombre d'échantillons.
            \end{itemize}
        \end{column}
    \end{columns}

\end{frame}
    



\section{Outils de paralélisation}

\begin{frame}
    \frametitle{Outils de paralélisation}
    \tableofcontents[currentsection]
\end{frame}

\subsection{Open MP}
\begin{frame}
    \frametitle{Open MP}
    \small
    \begin{itemize}
        \item \textbf{Intégration de Gauss 2D }
            \begin{itemize}
                \item Parallélisation de la génération de points et de poids avec \\\texttt{\#pragma omp parallel for collapse(2)}.
                \item L'utilisation de \texttt{collapse(2)} exploite le parallélisme dans les deux boucles imbriquées, améliorant l'efficacité.
            \end{itemize}
        \item \textbf{Intégration de Monte Carlo }
            \begin{itemize}
                \item La boucle de génération de points aléatoires est parallélisée avec \texttt{\#pragma omp parallel for reduction(+:total)}.
                \item Chaque thread utilise une graine différente (\texttt{rd() + thread\_id}) pour éviter les séquences aléatoires identiques.
            \end{itemize}
        \item \textbf{Runge-Kutta }
            \begin{itemize}
                \item Boucle temporelle parallélisée avec \texttt{\#pragma omp parallel for}.
                \item Copie temporaire des données (\texttt{std::vector<double> tempU(u)}) utilisée pour chaque thread, évitant les dépendances de données et optimisant les calculs.
            \end{itemize}
        \item \textbf{Intégration de Simpson}
            \begin{itemize}
                \item Boucles \texttt{for} parallélisées avec \texttt{\#pragma omp parallel for reduction(+:integral)} pour les deux versions.
                \item Pour \texttt{compositeSimpsons\_3\_8}, utilisation d'un pas de 3 pour les indices de boucle, améliorant l'efficacité de la méthode.
            \end{itemize}
    \end{itemize}
\end{frame}



\begin{frame}
    \frametitle{MPI}
    \small
    \begin{itemize}
        \item \textbf{Intégration Gauss 2D}
            \begin{itemize}
                \item Parallélisation de la génération de points et de poids.
                \item Utilisation de MPI\_Sendrecv pour l'échange des bords entre les processus.
            \end{itemize}
        \item \textbf{Intégration Monte Carlo }
            \begin{itemize}
                \item Division équitable des points entre les processus.
                \item Utilisation de MPI\_Reduce pour obtenir le résultat global.
            \end{itemize}
        \item \textbf{Runge-Kutta}
            \begin{itemize}
                \item Division des données spatiales entre les processus.
                \item Utilisation de MPI\_Gather pour collecter les résultats.
            \end{itemize}
        \item \textbf{Intégration Simpson}
            \begin{itemize}
                \item Division équitable des points entre les processus.
                \item Utilisation de MPI\_Reduce pour obtenir le résultat global.
            \end{itemize}
    \end{itemize}
\end{frame}


  
  

\subsection{CUDA}
\begin{frame}
    \frametitle{CUDA}
\end{frame}

\section{Résultats}

\begin{frame}
    \frametitle{Résultats}
    \tableofcontents[currentsection]
\end{frame}

\begin{frame}
    \frametitle{Résultats pour Open MP}
        \small
    \begin{tabular}{cc}
        \includegraphics[width=0.45\linewidth]{../Images/time_gauss_Op_MP.png} &
        \includegraphics[width=0.45\linewidth]{../Images/time_simp_Op_MP.png} \\
        Gauss 2D & Simpson \\
        \includegraphics[width=0.45\linewidth]{../Images/time_RK_Op_MP.png} &
        \includegraphics[width=0.45\linewidth]{../Images/time_montecarlo_Op_MP.png} \\
        Runge Kutta & Monte Carlo \\
    \end{tabular}
        
        % \begin{figure}
        %     \begin{minipage}{0.4\linewidth}
        %         \includegraphics[width=\linewidth]{../Images/time_gauss_Op_MP.png}
        %         \caption{Gauss 2D}
        %     \end{minipage}\hfill
        %     \begin{minipage}{0.4\linewidth}
        %         \includegraphics[width=\linewidth]{../Images/time_simp_Op_MP.png}
        %         \caption{Simpson}
        %     \end{minipage}
        % \end{figure}
    
        % \begin{figure}
        %     \begin{minipage}{0.4\linewidth}
        %         \includegraphics[width=\linewidth]{../Images/time_RK_Op_MP.png}
        %         \caption{Runge Kutta}
        %     \end{minipage}\hfill
        %     \begin{minipage}{0.4\linewidth}
        %         \includegraphics[width=\linewidth]{../Images/time_montecarlo_Op_MP.png}
        %         \caption{Monte Carlo}
        %     \end{minipage}
        % \end{figure}

\end{frame}


\subsection{Résultats pour MPI}

\begin{frame}
    \frametitle{MPI}
        \small
    \begin{tabular}{cc}
        \includegraphics[width=0.45\linewidth]{../Images/time_gauss_MPI.png} &
        \includegraphics[width=0.45\linewidth]{../Images/time_simp_MPI.png} \\
        Gauss 2D & Simpson \\
        \includegraphics[width=0.45\linewidth]{../Images/time_RK_MPI.png} &
        \includegraphics[width=0.45\linewidth]{../Images/time_montecarlo_MPI.png} \\
        Runge Kutta & Monte Carlo \\
    \end{tabular}
        
\end{frame}

\subsection{Résultats pour CUDA}

\begin{frame}
    \frametitle{CUDA}
        \small
    \begin{tabular}{cc}
        \includegraphics[width=0.45\linewidth]{../Images/time_simpson_cuda.png} &
        \includegraphics[width=0.45\linewidth]{../Images/time_Gauss2D_cuda.png} \\
        Gauss 2D & Simpson \\
        % \includegraphics[width=0.45\linewidth]{../Images/time_RK_cuda.png} &
        % \includegraphics[width=0.45\linewidth]{../Images/time_montecarlo_cuda.png} \\
        Runge Kutta & Monte Carlo \\
    \end{tabular}
        
\end{frame}

\end{document}